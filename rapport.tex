%%%%%%%%%%%%%%%%%%%%%%%%%%%%%%%%%%%%%%%%%%%%%%%%%%%%%%%%%%%%%
%% Based on a TeXnicCenter-Template, which was             %%
%% created by Christoph Börensen                           %%
%% and slightly modified by Tino Weinkauf.                 %%
%%                                                         %%
%% Then, a third guy - me - put in some modifications.     %%
%%                                                         %%
%% IFT2245 - Rapport TP1                                   %%
%%%%%%%%%%%%%%%%%%%%%%%%%%%%%%%%%%%%%%%%%%%%%%%%%%%%%%%%%%%%%

\documentclass[letterpaper,12pt]{scrartcl}
% Optimised for letter. Add ",twosides" to use the two-sides layout.

% Margins
    \usepackage{vmargin}
    \setpapersize{USletter}
    \setmargins{2.0cm}%	 % Left edge
               {1.5cm}%  % Top edge
               {17.7cm}% % Text width
               {21.0cm}% % Text height
               {14pt}%	 % Header height
               {1cm}%    % Header distance
               {0pt}%	 % Footer height
               {2cm}%    % Footer distance
				
% Graphical bugfix (about footnotes)
    \usepackage[bottom]{footmisc}

% Fonts and locale
	\usepackage{t1enc}
	\usepackage[utf8]{inputenc}
	\usepackage{times}
	\usepackage[francais]{babel}
	\usepackage{amsmath}

	\AtBeginDocument {%
	    \renewcommand\tablename{\textsc{Tableau}}
	}

% Graphics
	\usepackage[pdftex]{graphicx}
	\usepackage{color}
	\usepackage{eso-pic}
	\usepackage{everyshi}
	\renewcommand{\floatpagefraction}{0.7}

% Enable hyperlinks
	\usepackage[pdftex=true]{hyperref}
	
% Table layout
	\usepackage{booktabs}

% Caption
	\usepackage{ccaption}
	\captionnamefont{\bf\footnotesize\sffamily}
	\captiontitlefont{\footnotesize\sffamily}
	\setlength{\abovecaptionskip}{0mm}

% Header and footer settings
	\usepackage{scrpage2} 
	\renewcommand{\headfont}{\footnotesize\sffamily}
	\renewcommand{\pnumfont}{\footnotesize\sffamily}

% Pagestyles
	\defpagestyle{cb}{
		(\textwidth,0pt) % Sets the border line above the header
		{\pagemark\hfill\headmark\hfill} % Doublesided, left page
		{\hfill\headmark\hfill\pagemark} % Doublesided, right page
		{\hfill\headmark\hfill\pagemark} % Onesided
		(\textwidth,1pt)} % Sets the border line below the header
		{(\textwidth,1pt) % Sets the border line above the footer
		{{\it Rapport TP1 (IFT2245)}\hfill Sulliman Aïad et François Poitras} % Doublesided, left page
		{Charles Langlois et François Poitras\hfill{\it Rapport TP1 (IFT2245)}} % Doublesided, right page
		{Charles Langlois et François Poitras\hfill{\it Rapport TP1 (IFT2245)}} % One sided printing
		(\textwidth,0pt) % Sets the border line below the footer
	}

% Empty pages style
	\renewpagestyle{plain}
		{(\textwidth,0pt)
			{\hfill}{\hfill}{\hfill}
		(\textwidth,0pt)}
		{(\textwidth,0pt)
			{\hfill}{\hfill}{\hfill}
		(\textwidth,0pt)}

% Footnotes
	\renewcommand{\footnoterule}{\rule{5cm}{0.2mm} \vspace{0.3cm}}
	\deffootnote[1em]{1em}{1em}{\textsuperscript{\normalfont\thefootnotemark}}

\pagestyle{plain}

\begin{document}
	\begin{center}
		\vspace{2cm}

		{\Huge\bf\sf Rapport du Travail Pratique 2}

		\vspace{0.5cm}

		{\bf\sf (TP1)}

		\vspace{4cm}

		{\bf\sf Par}

		\vspace{0.5cm}{\large\bf\sf Charles Langlois et François Poitras}

		\vspace{2cm}

		{\bf\sf Rapport présenté à}

		\vspace{0.5cm}{\large\bf\sf M. Stefan Monnier}

		\vspace{2cm}

		{\bf\sf Dans le cadre du cours de}

		\vspace{0.5cm}{\large\bf\sf Systèmes d'exploitation (IFT2245)}

		\vspace{\fill}
		\today

		\vspace{0.5cm}Université de Montréal
	\end{center}
	
	\newpage

	\pagestyle{cb}
	
	\tableofcontents

	\newpage
\section{Fonctionnement du programme}
Le serveur commence en attedant la requête initiale, encodée sous forme d'énumération comme un \textit{BEGIN}. Cette requête est aussi constituée du nombre de clients et du nombre de ressources que chaque client peut utiliser. Une fois que le serveur possède ces informations, il peut initialiser les structures de données requises au bon fonctionnement de l'algorithme du banquier, décrit plus dans détail dans la section sur la gestion des inter-blocages. Le serveur initialise par la suite un \textit{thread} par client, pour gérer les connexions. Cette approche, différente de celle proposée dans l'énoncé du travail, est beaucoup plus simple. En effet, nous n'avons pas à gérer de file d'attente entre différents clients qui essairaient de se connecter sur un même socket, dans un même thread.

Une fois l'initialisation des structures terminée, le programme du client commence à envoyer ses requêtes, définies comme variables de configuration. La quantité de ressource à utiliser est déterminée de manière aléatoire avec un appel à \textit{rand(3)}. Remarquons que cette fonction retourne toujours la même suite d'entiers si elle est appelée dans les mêmes conditions. Pour éviter ce problème, nous commençons à un endroit pseudo-aléatoire dans la suite, à l'aide de la fonction \textit{srand(3)} et du temps courant. La fonction de génération de nombre aléatoires est aussi utilisée pour déterminer si le client doit tenter de libérer ou de prendre des ressources. Pour déterminer ce paramètre, nous vérifions la parité du nombre généré. 

Lorsque toutes les données nécéssaires à la construction d'uen requête sont complètes, celle-ci est envoyée à l'aide du \textit{socket} approprié. Notons que le client envoie toujours (nb_ressources+2) entiers. Dans la configuration de base, il y a trois ressources, donc le client envoie cinq entiers. Les deux autres sont respectivement le code du message et un nombre relatif au code envoyé. Par exemple, dans le  Il est possible que l'opération ne réussise pas d'un coup, si le système est particulièrement lent ou congestionné. C'est pourquoi les opérations de lecture et d'écritures sont dans des boucles et utilisent de l'artithmétique de pointeurs pour positionner correctement l'information reçue ou transmise. Les données sont d'abord lues dans un \textit{buffer} qui est en fait l'adresse mémoire où il faut stocker ce qui est lu ou écrit.
\section{Gestion des deadlocks}
\section{Prévention de la corruption}
\section{Synchronisation de la fin de l'éxecution}
Le serveur est programmé de manière à ne pas se terminer tant qu'il reste au moins un client d'actif ou tant que le temps limite n'est pas atteint. Si tous les clients sont terminés, alors le serveur se termine en appelant les fonctions de fin de traitement de sockets, telles que \textit{close(2)} et \textit{shutdown(2)}. Dans le cas où le \textit{timeout} serait atteint, nous fermons tous les threads serveurs et la fonction de \textit{shutdown} est appelée avec le paramètre \textit{SHUT\_RDWR}, qui a pour effet de faire un \textit{flush} sur les données qui pourraient êtres en cours de transfert. Cela fait que les clients peuvent par la suite arrêter d'envoyer des données sans causer d'erreur.

Si une erreur de lecture ou d'écriture survient, par exemple si le serveur et un client perdent leur connexion, le programme qui détecte l'erreur (à l'aide du code de retour) arrête tout traitement et termine. Cela va enchaîner une erreur de lecture dans l'autre programme, qui va arrêter à son tour.
%% ¡¡ REMPLIR ICI !!

\end{document}
